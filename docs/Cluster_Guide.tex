% Options for packages loaded elsewhere
\PassOptionsToPackage{unicode}{hyperref}
\PassOptionsToPackage{hyphens}{url}
%
\documentclass[
]{book}
\usepackage{amsmath,amssymb}
\usepackage{lmodern}
\usepackage{iftex}
\ifPDFTeX
  \usepackage[T1]{fontenc}
  \usepackage[utf8]{inputenc}
  \usepackage{textcomp} % provide euro and other symbols
\else % if luatex or xetex
  \usepackage{unicode-math}
  \defaultfontfeatures{Scale=MatchLowercase}
  \defaultfontfeatures[\rmfamily]{Ligatures=TeX,Scale=1}
\fi
% Use upquote if available, for straight quotes in verbatim environments
\IfFileExists{upquote.sty}{\usepackage{upquote}}{}
\IfFileExists{microtype.sty}{% use microtype if available
  \usepackage[]{microtype}
  \UseMicrotypeSet[protrusion]{basicmath} % disable protrusion for tt fonts
}{}
\makeatletter
\@ifundefined{KOMAClassName}{% if non-KOMA class
  \IfFileExists{parskip.sty}{%
    \usepackage{parskip}
  }{% else
    \setlength{\parindent}{0pt}
    \setlength{\parskip}{6pt plus 2pt minus 1pt}}
}{% if KOMA class
  \KOMAoptions{parskip=half}}
\makeatother
\usepackage{xcolor}
\IfFileExists{xurl.sty}{\usepackage{xurl}}{} % add URL line breaks if available
\IfFileExists{bookmark.sty}{\usepackage{bookmark}}{\usepackage{hyperref}}
\hypersetup{
  pdftitle={FH Cluster Guide},
  hidelinks,
  pdfcreator={LaTeX via pandoc}}
\urlstyle{same} % disable monospaced font for URLs
\usepackage{longtable,booktabs,array}
\usepackage{calc} % for calculating minipage widths
% Correct order of tables after \paragraph or \subparagraph
\usepackage{etoolbox}
\makeatletter
\patchcmd\longtable{\par}{\if@noskipsec\mbox{}\fi\par}{}{}
\makeatother
% Allow footnotes in longtable head/foot
\IfFileExists{footnotehyper.sty}{\usepackage{footnotehyper}}{\usepackage{footnote}}
\makesavenoteenv{longtable}
\usepackage{graphicx}
\makeatletter
\def\maxwidth{\ifdim\Gin@nat@width>\linewidth\linewidth\else\Gin@nat@width\fi}
\def\maxheight{\ifdim\Gin@nat@height>\textheight\textheight\else\Gin@nat@height\fi}
\makeatother
% Scale images if necessary, so that they will not overflow the page
% margins by default, and it is still possible to overwrite the defaults
% using explicit options in \includegraphics[width, height, ...]{}
\setkeys{Gin}{width=\maxwidth,height=\maxheight,keepaspectratio}
% Set default figure placement to htbp
\makeatletter
\def\fps@figure{htbp}
\makeatother
\setlength{\emergencystretch}{3em} % prevent overfull lines
\providecommand{\tightlist}{%
  \setlength{\itemsep}{0pt}\setlength{\parskip}{0pt}}
\setcounter{secnumdepth}{5}
\ifLuaTeX
  \usepackage{selnolig}  % disable illegal ligatures
\fi

\title{FH Cluster Guide}
\author{}
\date{\vspace{-2.5em}September 09, 2022}

\begin{document}
\maketitle

{
\setcounter{tocdepth}{1}
\tableofcontents
}
\hypertarget{about-this-course}{%
\chapter*{About this Course}\label{about-this-course}}
\addcontentsline{toc}{chapter}{About this Course}

Our goal is to get you running on the Fred Hutch cluster quickly and efficiently with this quick-start guide. As a wise Drivers' Ed instructor once said, \textbf{you need to go slow to go fast}!

In this short course, you'll invest a bit of time now to save you time and frustration down the road. Follow along at \textbf{any time on your own schedule}. We hope that the following modules will help you take advantage of the powerful resources the Fred Hutch has to offer!

\hypertarget{part-cluster-101}{%
\part*{Cluster 101}\label{part-cluster-101}}
\addcontentsline{toc}{part}{Cluster 101}

\hypertarget{what-is-a-cluster}{%
\chapter{What is a Cluster?}\label{what-is-a-cluster}}

A computing cluster is a set of many computers networked together. Because there are many computers working together, the network is able to handle computationally expensive tasks, like genome assemblies or advanced algorithms. Imagine you're building a house. It would take a long time by yourself! It's much better to have many builders working together.

Now that we have a team of workers, the next challenge is task management. A home construction team will need a manager to help delegate tasks. Similarly, the computing cluster uses management software to prioritize tasks, delegate workers (resources), and check on progress. The Fred Hutch cluster uses a common management and scheduling tool called \href{https://slurm.schedmd.com/overview.html}{Slurm}.

\includegraphics[width=1\linewidth]{resources/images/01-101_files/figure-latex//1BQxrVYdKZTbpCaF-i_q9w7s9x034lEXpQZDU-Sl09cs_g149d37dd4a1_0_18}

How is the cluster different from a laptop or desktop? First, you might use an operating system like Windows or MacOS. The Fred Hutch server is a Linux system. Second, because many people use the cluster for many tasks, there isn't a central screen and keyboard. You access the cluster remotely from your computer! We will talk more about how to connect to the cluster in a \protect\hyperlink{terminal}{following chapter}.

\textbf{Computing cluster}

A set of computers networked together to perform large tasks.

\hypertarget{account-setup}{%
\chapter{Account Setup}\label{account-setup}}

You will need an account to log in to the cluster. This ensures that data stays protected.

\hypertarget{check-your-hutchnet-id}{%
\section{Check your HutchNet ID}\label{check-your-hutchnet-id}}

A \href{https://centernet.fredhutch.org/cn/u/center-it/help-desk.html}{HutchNet ID} is the standard login you receive when you start working at the Hutch or are an official affiliate. You can use it to login to most resources at the Center (Desktop Computer, Employee Self Service, VPN, Webmail) and our Scientific Computing systems.

For example:

\begin{itemize}
\tightlist
\item
  my email is \texttt{jsmith3@fredhutch.org}.\\
\item
  my HutchNet ID is \texttt{jsmith3}.
\end{itemize}

If one of your collaborators requires access to the Fred Hutch network you can submit a \href{https://centernet.fredhutch.org/cn/f/hr/lcex/non-employee-action-form.html}{non-employee action form}. Non-employee is a generic administrative term for affiliates, students, contractors, etc.

\hypertarget{contacting-the-scicomp-team}{%
\section{Contacting the SciComp Team}\label{contacting-the-scicomp-team}}

To use Scientific Computing clusters at Hutch, your HutchNet ID must be associated with a PI account.

The Scientific Computing Team (SciComp) tries to set some users up ahead of time. However, not everyone is set up automatically. Please fill out \href{https://forms.gle/5ct8mQCeBD7LUt6S7}{this Account Setup Form} and we will ensure you are set up correctly!

Errors similar to ``Invalid account or account/partition'' typically indicate that the account hasn't been set up by SciComp. This is a quick fix if you use the form above.

Now, let's set up our Terminal!

\hypertarget{terminal}{%
\chapter{Terminal Setup}\label{terminal}}

The next step is getting familiar with your Terminal. This is your portal to the cluster.

\hypertarget{what-is-a-terminal}{%
\section{What is a terminal?}\label{what-is-a-terminal}}

The Terminal is a \href{https://www.codecademy.com/article/command-line-interface}{command line interface}. In other words, the Terminal is a software application that allows you to issue commands directly to your laptop or desktop computer. The Terminal is very useful because it allows you to run commands that don't have a graphical user interface (GUI). It can also connect you to computer networks, such as the Fred Hutch cluster! The Terminal setup is different depending on your operating system. Jump to the \protect\hyperlink{windows}{Windows}, \protect\hyperlink{mac}{MacOS}, or \protect\hyperlink{linux}{Linux} sections below.

``Terminal'' used to be synonymous with ``computer''. With the creation of operating systems like Windows and MacOS, computers became much easier to use and exploded in popularity! Your colleagues are almost always referring to the command line application when they say ``Terminal''.

\hypertarget{windows}{%
\section{Windows Setup}\label{windows}}

Click to view steps

You will need to install a Terminal application called PuTTY to connect to the Fred Hutch Cluster.

\begin{enumerate}
\def\labelenumi{\arabic{enumi}.}
\item
  You should then see PuTTY available in the Software Center. Click ``Install'' and go through the Setup Wizard.

  \includegraphics[width=1\linewidth]{resources/images/01-101_files/figure-latex//1BQxrVYdKZTbpCaF-i_q9w7s9x034lEXpQZDU-Sl09cs_g15643d101eb_4_15}

  \includegraphics[width=1\linewidth]{resources/images/01-101_files/figure-latex//1BQxrVYdKZTbpCaF-i_q9w7s9x034lEXpQZDU-Sl09cs_g15643d101eb_4_20}
\end{enumerate}

You can also \href{faq.html\#manual-putty}{install PuTTY manually} if you don't see it in the Software Center.

\begin{enumerate}
\def\labelenumi{\arabic{enumi}.}
\item
  PuTTY should now be available in your applications. Click on PuTTY to open.

  \includegraphics[width=1\linewidth]{resources/images/01-101_files/figure-latex//1BQxrVYdKZTbpCaF-i_q9w7s9x034lEXpQZDU-Sl09cs_g15643d101eb_4_28}
\item
  You should now see the PuTTY Configuration menu.

  \includegraphics[width=1\linewidth]{resources/images/01-101_files/figure-latex//1BQxrVYdKZTbpCaF-i_q9w7s9x034lEXpQZDU-Sl09cs_g15643d101eb_4_35}
\end{enumerate}

\hypertarget{mac}{%
\section{Mac Setup}\label{mac}}

Click to view steps

Mac machines come with a Terminal installed.

\begin{enumerate}
\def\labelenumi{\arabic{enumi}.}
\item
  Go to Finder \textgreater{} Applications \textgreater{} Utilities \textgreater{} Terminal and double-click.

  \includegraphics[width=1\linewidth]{resources/images/01-101_files/figure-latex//1BQxrVYdKZTbpCaF-i_q9w7s9x034lEXpQZDU-Sl09cs_g149d37dd4a1_0_9}
\item
  Your Terminal should look like this:

  \includegraphics[width=1\linewidth]{resources/images/01-101_files/figure-latex//1BQxrVYdKZTbpCaF-i_q9w7s9x034lEXpQZDU-Sl09cs_g149d37dd4a1_0_2}
\end{enumerate}

\hypertarget{linux}{%
\section{Linux Setup}\label{linux}}

Click to view steps

The commonly used Linux distribution, Ubuntu, already comes with a Terminal installed.

\begin{enumerate}
\def\labelenumi{\arabic{enumi}.}
\tightlist
\item
  Press ctrl + alt + T. Your open Terminal window should look like this:
\end{enumerate}

{[}SCREENSHOT{]}

\begin{enumerate}
\def\labelenumi{\arabic{enumi}.}
\tightlist
\item
  Update the Terminal and prepare it for connecting to the cluster by running:
\end{enumerate}

\begin{verbatim}
sudo apt install openssh-server
\end{verbatim}

Enter your password and enter \texttt{Y} when prompted.

\hypertarget{logging-in}{%
\chapter{Logging In}\label{logging-in}}

Now that you have your Terminal application ready, you want to connect to the cluster. You will do this using a method called \href{https://www.ssh.com/academy/ssh/protocol}{SSH}, which stands for ``Secure SHell''.

\hypertarget{what-is-ssh}{%
\section{\texorpdfstring{What is \texttt{SSH}?}{What is SSH?}}\label{what-is-ssh}}

SSH is a secure way to remotely connect to another computer or network of computers. In other words, SSH helps us protect your data and the data on the Fred Hutch cluster through authentication.

Before moving on, you will need to connect to the \href{https://centernet.fredhutch.org/cn/u/center-it/help-desk/vpn.html}{Fred Hutch VPN}. This is the first layer of security. The next set of steps are specific to your operating system.

\hypertarget{windows-login}{%
\section{Windows Login}\label{windows-login}}

Click to view steps

\begin{enumerate}
\def\labelenumi{\arabic{enumi}.}
\item
  Go to the PuTTY Configuration menu. Under ``Host Name'' type \textbf{rhino} and click ``Open''.

  \includegraphics[width=1\linewidth]{resources/images/01-101_files/figure-latex//1BQxrVYdKZTbpCaF-i_q9w7s9x034lEXpQZDU-Sl09cs_g15643d101eb_4_41}
\item
  You will be prompted to login. Type in your HutchNetID (e.g., \texttt{jsmith3}).

  \includegraphics[width=1\linewidth]{resources/images/01-101_files/figure-latex//1BQxrVYdKZTbpCaF-i_q9w7s9x034lEXpQZDU-Sl09cs_g15643d101eb_4_48}
\item
  Enter your password. No\texttt{*} or symbols will show up, so type it in carefully!
\item
  You are now logged in! There should be a login message, with your name at the bottom.

  \includegraphics[width=1\linewidth]{resources/images/01-101_files/figure-latex//1BQxrVYdKZTbpCaF-i_q9w7s9x034lEXpQZDU-Sl09cs_g15643d101eb_4_60}
\end{enumerate}

Congratulations! You are now logged in to the Fred Hutch cluster!

\hypertarget{mac-login}{%
\section{Mac Login}\label{mac-login}}

Click to view steps

\begin{enumerate}
\def\labelenumi{\arabic{enumi}.}
\item
  Type the following commands, substituting in your HutchNet ID:

\begin{verbatim}
ssh -Y HutchID@rhino
\end{verbatim}
\item
  You will see a message that looks like \texttt{The\ authenticity\ of\ host\ \textquotesingle{}rhino\ (XXX.XXX.XX.XX)\textquotesingle{}\ can\textquotesingle{}t\ be\ established.} Type in \texttt{yes} and hit enter.
\item
  Enter your password. No\texttt{*} or symbols will show up, so type it in carefully!
\item
  You are now logged in! There should be a login message, with your name at the bottom.

  \includegraphics[width=1\linewidth]{resources/images/01-101_files/figure-latex//1BQxrVYdKZTbpCaF-i_q9w7s9x034lEXpQZDU-Sl09cs_g149d37dd4a1_0_43}
\end{enumerate}

Congratulations! You are now logged in to the Fred Hutch cluster!

\hypertarget{linux-login}{%
\section{Linux Login}\label{linux-login}}

Click to view steps

Congratulations! You are now logged in to the Fred Hutch cluster!

\hypertarget{tour-your-space}{%
\chapter{Tour your space}\label{tour-your-space}}

\hypertarget{submit-your-first-job}{%
\chapter{Submit your first job}\label{submit-your-first-job}}

\hypertarget{interactive-cluster}{%
\chapter{Interactive cluster}\label{interactive-cluster}}

\hypertarget{file-upload-and-download}{%
\chapter{File upload and download}\label{file-upload-and-download}}

\hypertarget{globus}{%
\section{Globus}\label{globus}}

\hypertarget{part-cluster-201}{%
\part*{Cluster 201}\label{part-cluster-201}}
\addcontentsline{toc}{part}{Cluster 201}

\hypertarget{resource-optimization}{%
\chapter{Resource Optimization}\label{resource-optimization}}

It is best practice to\ldots{}

\hypertarget{part-appendix}{%
\part*{Appendix}\label{part-appendix}}
\addcontentsline{toc}{part}{Appendix}

\hypertarget{help}{%
\chapter{Where to get help}\label{help}}

We want to help! Here are some ways you can get help for your work on the cluster.

\hypertarget{submit-a-ticket}{%
\section*{Submit a Ticket}\label{submit-a-ticket}}
\addcontentsline{toc}{section}{Submit a Ticket}

Submitting a good ticket helps the SciComp Team address your needs quickly and efficiently. We suggest you submit the following in a ticket:

\begin{enumerate}
\def\labelenumi{\arabic{enumi}.}
\tightlist
\item
\item
\item
\end{enumerate}

\hypertarget{visit-the-sciwiki}{%
\section*{Visit the SciWiki}\label{visit-the-sciwiki}}
\addcontentsline{toc}{section}{Visit the SciWiki}

The SciWiki \href{https://sciwiki.fredhutch.org/scicomputing/comp_index/}{Scientific Computing page} is full of useful tips and guides.

\hypertarget{feedback}{%
\chapter{Provide Feedback}\label{feedback}}

Please submit an issue at our \href{https://github.com/fhdsl/FH_Cluster_Guide/issues/new}{GitHub repo}. You can also click the edit button on the top of the page in question.

\hypertarget{faq}{%
\chapter*{FAQ}\label{faq}}
\addcontentsline{toc}{chapter}{FAQ}

Here are some issues you might encounter.

\hypertarget{manual-putty}{%
\section{How can I manually install PuTTY?}\label{manual-putty}}

Click to view steps

\begin{enumerate}
\def\labelenumi{\arabic{enumi}.}
\item
  Click \href{https://www.chiark.greenend.org.uk/~sgtatham/putty/latest.html}{here} to install the latest version of PuTTY. You will choose the 64-bit x86 installation with few exceptions.

  \includegraphics[width=1\linewidth]{faq_files/figure-latex//1BQxrVYdKZTbpCaF-i_q9w7s9x034lEXpQZDU-Sl09cs_g15643d101eb_4_6}
\item
  Click through to install via the Setup Wizard.
\end{enumerate}

\hypertarget{about-the-authors}{%
\chapter*{About the Authors}\label{about-the-authors}}
\addcontentsline{toc}{chapter}{About the Authors}

These credits are based on our \href{https://github.com/jhudsl/OTTR_Template/wiki/How-to-give-credits}{course contributors table guidelines}.

~
~

\begin{longtable}[]{@{}
  >{\raggedright\arraybackslash}p{(\columnwidth - 2\tabcolsep) * \real{0.58}}
  >{\raggedright\arraybackslash}p{(\columnwidth - 2\tabcolsep) * \real{0.42}}@{}}
\toprule
\begin{minipage}[b]{\linewidth}\raggedright
Credits
\end{minipage} & \begin{minipage}[b]{\linewidth}\raggedright
Names
\end{minipage} \\
\midrule
\endhead
\textbf{Pedagogy} & \\
Lead Content Instructor(s) & \href{link\%20to\%20personal\%20website}{FirstName LastName} \\
Lecturer(s) (include chapter name/link in parentheses if only for specific chapters) - make new line if more than one chapter involved & Delivered the course in some way - video or audio \\
Content Author(s) (include chapter name/link in parentheses if only for specific chapters) - make new line if more than one chapter involved & If any other authors besides lead instructor \\
Content Contributor(s) (include section name/link in parentheses) - make new line if more than one section involved & Wrote less than a chapter \\
Content Editor(s)/Reviewer(s) & Checked your content \\
Content Director(s) & Helped guide the content direction \\
Content Consultants (include chapter name/link in parentheses or word ``General'') - make new line if more than one chapter involved & Gave high level advice on content \\
Acknowledgments & Gave small assistance to content but not to the level of consulting \\
\textbf{Production} & \\
Content Publisher(s) & Helped with publishing platform \\
Content Publishing Reviewer(s) & Reviewed overall content and aesthetics on publishing platform \\
\textbf{Technical} & \\
Course Publishing Engineer(s) & Helped with the code for the technical aspects related to the specific course generation \\
Template Publishing Engineers & \href{https://www.cansavvy.com/}{Candace Savonen}, \href{https://carriewright11.github.io/}{Carrie Wright} \\
Publishing Maintenance Engineer & \href{https://www.cansavvy.com/}{Candace Savonen} \\
Technical Publishing Stylists & \href{https://carriewright11.github.io/}{Carrie Wright}, \href{https://www.cansavvy.com/}{Candace Savonen} \\
Package Developers (\href{https://github.com/jhudsl/ottrpal}{ottrpal}) \href{https://www.cansavvy.com/}{Candace Savonen}, \href{https://johnmuschelli.com/}{John Muschelli}, \href{https://carriewright11.github.io/}{Carrie Wright} & \\
\textbf{Art and Design} & \\
Illustrator(s) & Created graphics for the course \\
Figure Artist(s) & Created figures/plots for course \\
Videographer(s) & Filmed videos \\
Videography Editor(s) & Edited film \\
Audiographer(s) & Recorded audio \\
Audiography Editor(s) & Edited audio recordings \\
\textbf{Funding} & \\
Funder(s) & Institution/individual who funded course including grant number \\
Funding Staff & Staff members who help with funding \\
\bottomrule
\end{longtable}

~

\begin{verbatim}
## - Session info ---------------------------------------------------------------
##  setting  value                       
##  version  R version 4.0.2 (2020-06-22)
##  os       Ubuntu 20.04.3 LTS          
##  system   x86_64, linux-gnu           
##  ui       X11                         
##  language (EN)                        
##  collate  en_US.UTF-8                 
##  ctype    en_US.UTF-8                 
##  tz       Etc/UTC                     
##  date     2022-09-09                  
## 
## - Packages -------------------------------------------------------------------
##  package     * version    date       lib source                            
##  assertthat    0.2.1      2019-03-21 [1] RSPM (R 4.0.3)                    
##  bookdown      0.24       2022-02-15 [1] Github (rstudio/bookdown@88bc4ea) 
##  callr         3.4.4      2020-09-07 [1] RSPM (R 4.0.2)                    
##  cli           2.0.2      2020-02-28 [1] RSPM (R 4.0.0)                    
##  crayon        1.3.4      2017-09-16 [1] RSPM (R 4.0.0)                    
##  desc          1.2.0      2018-05-01 [1] RSPM (R 4.0.3)                    
##  devtools      2.3.2      2020-09-18 [1] RSPM (R 4.0.3)                    
##  digest        0.6.25     2020-02-23 [1] RSPM (R 4.0.0)                    
##  ellipsis      0.3.1      2020-05-15 [1] RSPM (R 4.0.3)                    
##  evaluate      0.14       2019-05-28 [1] RSPM (R 4.0.3)                    
##  fansi         0.4.1      2020-01-08 [1] RSPM (R 4.0.0)                    
##  fs            1.5.0      2020-07-31 [1] RSPM (R 4.0.3)                    
##  glue          1.6.1      2022-01-22 [1] CRAN (R 4.0.2)                    
##  htmltools     0.5.0      2020-06-16 [1] RSPM (R 4.0.1)                    
##  knitr         1.33       2022-02-15 [1] Github (yihui/knitr@a1052d1)      
##  lifecycle     1.0.0      2021-02-15 [1] CRAN (R 4.0.2)                    
##  magrittr      2.0.2      2022-01-26 [1] CRAN (R 4.0.2)                    
##  memoise       1.1.0      2017-04-21 [1] RSPM (R 4.0.0)                    
##  pkgbuild      1.1.0      2020-07-13 [1] RSPM (R 4.0.2)                    
##  pkgload       1.1.0      2020-05-29 [1] RSPM (R 4.0.3)                    
##  prettyunits   1.1.1      2020-01-24 [1] RSPM (R 4.0.3)                    
##  processx      3.4.4      2020-09-03 [1] RSPM (R 4.0.2)                    
##  ps            1.3.4      2020-08-11 [1] RSPM (R 4.0.2)                    
##  purrr         0.3.4      2020-04-17 [1] RSPM (R 4.0.3)                    
##  R6            2.4.1      2019-11-12 [1] RSPM (R 4.0.0)                    
##  remotes       2.2.0      2020-07-21 [1] RSPM (R 4.0.3)                    
##  rlang         0.4.10     2022-02-15 [1] Github (r-lib/rlang@f0c9be5)      
##  rmarkdown     2.10       2022-02-15 [1] Github (rstudio/rmarkdown@02d3c25)
##  rprojroot     2.0.2      2020-11-15 [1] CRAN (R 4.0.2)                    
##  sessioninfo   1.1.1      2018-11-05 [1] RSPM (R 4.0.3)                    
##  stringi       1.5.3      2020-09-09 [1] RSPM (R 4.0.3)                    
##  stringr       1.4.0      2019-02-10 [1] RSPM (R 4.0.3)                    
##  testthat      3.0.1      2022-02-15 [1] Github (R-lib/testthat@e99155a)   
##  usethis       2.1.5.9000 2022-02-15 [1] Github (r-lib/usethis@57b109a)    
##  withr         2.3.0      2020-09-22 [1] RSPM (R 4.0.2)                    
##  xfun          0.32       2022-08-10 [1] CRAN (R 4.0.2)                    
##  yaml          2.2.1      2020-02-01 [1] RSPM (R 4.0.3)                    
## 
## [1] /usr/local/lib/R/site-library
## [2] /usr/local/lib/R/library
\end{verbatim}

\hypertarget{references}{%
\chapter*{References}\label{references}}
\addcontentsline{toc}{chapter}{References}

\end{document}
